\documentclass{article}

\usepackage[spanish]{babel}
\usepackage{amsmath}
\usepackage[utf8]{inputenc}

\title{MÉTODO SÍMPLEX}
\author{ChanclasMágicas}

\begin{document}
\maketitle
\tableofcontents

\section{Introducción}
\label{sec:introduccion}

El Método Simplex es un procedimiento iterativo que permite mejorar la
solución de la función objetivo en cada paso. El proceso concluye
cuando no es posible continuar mejorando dicho valor.
Dicho método surge en 1947 por el matemático norteamericano George
Dantzig.

\section{Ejemplo}
\label{sec:ejemplo}

\begin{equation*}
  \begin{aligned}
    \text{Maximizar}    \quad  2x_1-2x_2\\
    \text{sujeto a}    \quad &
    \begin{aligned}
      2x_1+x_2 & \leq 4 \\
      x_1+x_2 & \leq 5\\
      x_1,x_2 & \geq 0

    \end{aligned}
  \end{aligned}
\end{equation*}

El Tablero simplex queda como\\
\begin{equation*}
 
    \begin{aligned}
      x_3 & = 4-2x_1-x_2 \\
      x_4 & = 5-x_1-x_2\\

  \end{aligned}
\end{equation*}


\end{document}
